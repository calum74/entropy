\documentclass[12pt]{article}

% Packages
\usepackage[utf8]{inputenc}
\usepackage{amsmath, amssymb}   % Math
\usepackage{graphicx}           % Images
\usepackage{hyperref}           % Clickable links
\usepackage{geometry}           % Page margins
% \usepackage{cite}               % Citation formatting

% BibLaTeX for references (requires biber)
\usepackage[
    backend=biber,
    style=numeric,
    sorting=nyt
]{biblatex}

\addbibresource{references.bib}  % Bib file

% Page setup
\geometry{margin=1in}

% Title
\title{Algorithms for efficient entropy conversion}
\author{Calum Grant \\
OxFORD Asset Management \\
calum.grant@oxam.com}
\date{\today}

\begin{document}

\maketitle

\begin{abstract}
This is a short abstract summarizing the main results and contributions of the paper. Blah blah.
\end{abstract}

\section{Introduction}

This is the introduction to your paper. You can cite sources using \cite{shannon1948}.

\section{Mathematical Background}

Here is a sample equation:

\begin{equation}
H(p) = -p \log_2 p - (1 - p) \log_2(1 - p)
\label{eq:entropy}
\end{equation}

As shown in Equation~\ref{eq:entropy}, the binary entropy function measures the uncertainty of a Bernoulli variable.

\section{Methodology}

Describe your method here. Equations can be aligned:

\begin{align}
f(x) &= x^2 + 1 \\
f'(x) &= 2x
\end{align}

\section{Results}

Include results, tables, or figures:

\begin{figure}[h]
\centering
\includegraphics[width=0.5\textwidth]{example-image}
\caption{An example figure.}
\label{fig:example}
\end{figure}

\section{Conclusion}

Summarize your findings and potential future work.

\printbibliography

\end{document}
