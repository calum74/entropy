\documentclass[12pt]{article}

% Packages
\usepackage[utf8]{inputenc}
\usepackage{amsmath, amssymb, amsthm}   % Math
\usepackage{graphicx}           % Images
% \usepackage{hyperref}           % Clickable links
\usepackage{geometry}           % Page margins
% \usepackage{cite}               % Citation formatting
\usepackage{algorithm}
\usepackage{algpseudocode}
\usepackage{listings}
\usepackage{color}

\definecolor{dkgreen}{rgb}{0,0.6,0}
\definecolor{gray}{rgb}{0.5,0.5,0.5}
\definecolor{mauve}{rgb}{0.58,0,0.82}

% BibLaTeX for references (requires biber)
\usepackage[
    backend=biber,
    style=numeric,
    sorting=nyt
]{biblatex}

\addbibresource{references.bib}  % Bib file

% Page setup
\geometry{margin=1in}

% Title
\title{Efficient entropy conversion using an entropy store}
\author{Calum Grant \\
OxFORD Asset Management \\
calum.grant@oxam.com}
\date{\today}

\newtheorem{lemma}{Lemma}
\newtheorem{corollary}{Corollary}
\newtheorem{definition}{Definition}
\newtheorem{theorem}{Theorem}

\lstset{
  language=C,        % choose the language
  basicstyle=\ttfamily\small, % font style and size
  keywordstyle=\color{blue},
  commentstyle=\color{gray},
  stringstyle=\color{orange},
  numbers=left,
  numberstyle=\tiny\color{gray},
  stepnumber=1,
  numbersep=5pt,
  showstringspaces=false,
  breaklines=true,
  frame=single,
  captionpos=b
}

\begin{document}

\maketitle

\begin{abstract}
    We present new algorithms for converting random integers between different forms using an \em entropy store \em to cache unused entropy. The method can generate exact random variables for any weighted integer distribution, and consume entropy from any such distribution, whilst losing almost no entropy in the process.  For example we can shuffle a deck of 52 cards using just $\approxeq 225.58102$ bits of entropy, yielding an entropy conversion efficiency of $\approxeq 0.99999992$ using a 32-bit entropy cache, compared with classical algorithms that only have $\approxeq 0.81$ efficiency.  The algorithm requires two 32-bit integer divmod operations per random integer generated, making it slower than other methods.
\end{abstract}

\section{Introduction}

In this paper, we will study generating perfectly distributed integers from an entropy source. A typical example of this is using fair coin flips to roll a fair die or to perform a perfect shuffle of a deck of cards.

The general problem is one of \em entropy conversion \em, where entropy in one form needs to be converted to entropy in a different form using a function $f$ on a discrete random variable $X$ having Shannon entropy $H(X)$.  A fundamental result is that you cannot get more entropy out than you put in, or $H(X) \ge H(f(X))$ \ref{entropy}. We will be primarily focussed on improving the \em efficiency \em of entropy conversion, defined as $\eta = \frac{H(f(X))}{H(X)}$.

Whilst this problem has been studied extensively, there are theoretical limits when we don't store entropy, because algorithms must always fetch entropy in units. XX showed that we must fetch up to 2 extra bits of entropy per integer generated.  For example to roll a fair d6 die requires $\frac{11}{3} \approxeq 3.6$ coin flips on average, but the entropy contained in the d6 is $\log_26 \approxeq 2.58$. When shuffling a deck of 52 cards, fetching integers one at a time requires $\approxeq 278$ bits of entropy to perform the shuffle, against an output entropy of $\log_252! \approxeq 225.58$.

To mitigate these entropy losses, it's possible to generate random integers in batches. The major drawback with existing batching schemes is that they are somewhat complex and expensive to implement, or require you to know beforehand which numbers you need to generate. Whilst they improve entropy efficiency they still compromise on performance and efficiency.

In this paper we will explore a fundamentally different approach. We will allow every entropy conversion algorithm to have access to an \em entropy store \em in the form of a large uniformly distributed integer (of the order of 32 bits), and allow the algorithm to put back any unused entropy into the store. By carefully analysing the steps that could lose entropy, we can reclaim nearly all lost entropy, giving a near-perfect entropy conversion efficiency. 

The entropy efficiency depends only on the ratio of the size of the store $N$ to the size $n$ of the number generated, and a precise bound is calculated in Theorem \ref{xx}. For example to roll a 6-sided die with a 32-bit entropy store ($N\ge2^{31}$) has an entropy efficiency of $>0.99999997$ To shuffle a deck of 52 cards with a 32-bit entropy store has an entropy efficiency $>0.99999992$. By increasing the size of the store, we can achieve entropy efficiency arbitrarily close to 1.

The algorithms are easy to implement, require O(1) XX time and memory, and use two integer divmod operations per uniform integer generated. To convert between distributions, and additional $O(n)$ of lookup tables is required. The main benefit of these algorithms is their simplicity, flexibility and high efficiency.

\subsection {Contribution}

We present new algorithms, called \em entropy store algorithms \em, that have lower setup costs and higher amortised entropy efficiency than classical algorithms. Table \ref{tab:entropy-store} summarises the algorithm, where $m$ is the number of bits in each integer, and $n$ is the size of the input/output distribution. $\epsilon$ is defined in Theorem \ref{thm:loss} and graphed in Figure \ref{fig:uniform-losses}. These characteristics are essentially optimal since $m$ and $n$ are generally constant.  Of course, a weighted distribution also covers other distributions such as uniform and Bernoulli distributions.

\begin{table}[h!]
\centering
\begin{tabular}{|c|c|c|c|c|}
\hline
Input & Output & Entropy & Time & Space \\
\hline
Weighted & Weighted & $H(P)+\epsilon$ & Setup: $O(mn)$ & $O(mn)$ \\
Exact & Exact & (amortised) & Per output: $O(m \log m)$  &   \\
Markov & Markov &  &   &   \\
\hline
\end{tabular}
\caption{Entropy store algorithm overview}
    \label{tab:entropy-store}
\end{table}

No other algorithm has achieved this level of entropy efficiency in practice.

\subsection{Related work}

Entropy conversion is a well studied subject going back to at least 1951 when Von Neumann \cite{neumann51} proposed the first known algorithm for generating perfectly uniform integers (a "dice roll") from an unbiassed coin (a "coin flip") - just three years after Claude Shannon's seminal work from 1948 \cite{shannon1948mathematical}. Von Neumann's "rejection sampling" algorithm works by fetching the smallest power of 2 greater or equal to the number being generated. If the number is in range, return it, otherwise retry. This algorithm is elegant but not optimal in its entropy efficiency.

Knuth and Yao \cite{Knuth1976TheCO} came up with the first entropy efficient algorithm, and is based on creating a binary decision tree based on the arithmetic coding of each output, and proved that this solution is optimal in that no other binary decision tree could on average give fewer decisions.

The drawbacks with the original Knuth-Yao algorithm are the setup and space costs: the decision trees are explicitly constructed can get quite large. Lumbruso \cite{lumbroso2013optimal} devised an optimal dice rolling algorithm, that is of course no more efficient than Knuth-Yao, but requires no setup and is simple to implement. Bacher et al \cite{bacher2017} analyse the Von Neumann and Knuth-Yao algorithms from an entropy perspective, using Lumbruso's implementation, with a particular focus on generating unbiased permutations from unbiased coin flips. There are always up to 2 bits of entropy loss (or 'toll') for each uniform integer generated, depending on the integer is generated. Only exact powers of 2 can be generated without entropy loss.

Lemire \cite{lemire2019fast} compares different practical uniform generators, and describes a uniform generator that is efficient from a CPU perspective, but are less efficient from an entropy perspective, by using fewer CPU division operations.

Various methods have been explored for generating arbitrary intervals.
The Knuth-Yao algorithm can be used to generate arbitrary distributions from binary entropy, but the tree may need to be truncated unless the distribution is dyadic.

The alias method devised by Walker \cite{walker1977efficient} and improved by Vose \cite{vose91} maps a uniform integer to an arbitrary distribution, and is therefore less entropy efficient because the output distribution generally contains less entropy than the uniform input distribution.

Han and Hoshi \cite{han97} devised an optimal interval algorithm to convert biased or unbiased coin-flips to an arbitrary distribution, with an overhead of up to 3 bits of entropy per output.  The algorithm works by dividing the $[0,1)$ real interval according to the output distribution, and stops at bit n when the fetched binary number falls entirely within the output interval in the nth bit. Han and Hoshi show that this algorithm has an asymptotically optimal entropy consumption. 
Wanatabe \cite{wanatabe20} analyses this algorithm using an information spectrum, and Oohama \cite{oohama11, oohama2020performance} analyses the performance this algorithm.

/em Entropy extraction /em studies how to convert the entropy /em from /em a variety of different distributions. When the input distribution is unknown, Von Neumann \cite{neumann51} gives an algorithm to extract unbiased entropy from a biassed coin, which is again elegant but suboptimal in its entropy efficiency. Peres \cite{peres1992iterating} devised a way to recursively use the previously-discarded outputs to provide an algorithm with perfect amortised entropy efficiency, and the method was generalised by Pae \cite{pae15} to extract entropy from arbitrary but identically distributed unknown distributions ("weighted M-dice") with perfect amortised efficiency. Interestingly, it appears to be easier to extract entropy than to generate it.

For completely unknown distributions, Vembu et al \cite{vembu95} analyse the maximum rate at which entropy can be extracted. XX Implementation ?? XX

Batching is often cited as a way to achieve theoretical efficiency, by dividing the overhead of generating an output over the size of the output. Devroye \cite{devroye86} describes how to batch uniform distributions using multiplication, and the efficiency of tree-based batching in general.

The Knuth-Yau algorithm can be batched \cite{xx}.

FLDR batching: \cite{xx}

HanHoshi batching ...

Lumbroso \cite{lumbroso2013optimal} also suggests batching as a way to improve efficiency, but does not provide an implementation.

Bacher talk about using batching in random permutations. xx

Batching is an example of recycling unused entropy.

Ideas of recycling unused entropy are not new. Fill and Huber \cite{fill2000randomness} developed a Markov model which



Uniform generation:
    KnuthYao: \cite{Knuth1976TheCO}
    Lemire (low efficiency) \cite{lemire2019fast}
    Fast Loaded Dice roller: \cite{saad2020fldr}
    Arxiv: Lumbroso \cite{lumbroso2013optimal}
    Arxiv: Huber \cite{huber2024optimalrollingfairdice}
    Bacher: \cite{bacher2017}
    \cite{baidya24}
    \cite {von1963various}


Arbitrary distribution generation:
    \cite{walker1977efficient} (alias method)
    \cite{vose91}
    \cite{abrahams96}
    Interval generation:
        \cite{han97}
        \cite{watanabe2019}
        \cite{oohama11}
        \cite{oohama2020performance}
    Variate generation (not just weighted):
        \cite{saad2025}
    \cite{roche91}

Unclassified:
    \cite{norman1972computer}
    \cite{stout1984tree}
    \cite{Hoeffding1994}

Entropy extraction:
    \cite{elias1972efficient},
    Randomness recycler: @inproceedings{fill2000randomness}
    \cite{saad24}
    \cite{saad2020optimalsampling}
    \cite{saad2025}
    \cite{norman1972computer}
    \cite{pae15}
    \cite{pae20} ??
    \cite{peres1992iterating}
    \cite{pae06}
    \cite{vembu95}



\section{Algorithms for entropy conversion}

In this section we'll start with some basic operations (Algorithm \ref{alg:combine}-\ref{alg:resample}), then use these to create an efficient generator for uniform integers (Algorithm \ref{alg:generate-uniform}) using an entropy store.

We'll then build on Algorithm \ref{alg:generate-uniform} to create a generator for an arbitrary weighted distribution (Algorithm \ref{alg:generate-distribution}), extract entropy from a weighted distribution (Algorithm \ref{alg:combine-distribution}), from which we can build a general entropy converter. All of these algorithms are designed to have a 1 or near-1 entropy efficiency, and proofs of correctness and efficiency are contained throughout this section.

We'll use uppercase letters $X$, $Y$ and $Z$ for discrete random variables, and write $H(X)$ for the Shannon entropy of this variable. Write $X \sim Uniform\{0..n-1\}$ to mean that $X$ is distributed with a uniform distribution, or $X \sim Uniform\{n\}$ for short, because all distributions will be 0-based. $H(Uniform\{n\}) = \log_2n$. The interval $[a,b)$ means all integers in the range $a..b-1$.


\subsection{Fundamental operations}


Algorithm \ref{alg:combine} combines entropy from two uniform distrete random variables into a single uniform random variable.

\begin{algorithm}
\caption{Combining uniformly distributed integers}
\label{alg:combine}
\begin{algorithmic}[1]
    \Require $n$, $m$, $U_n$, $U_m$ are integers
    \Require $n>0$, $m>0$
    \Require $U_n$ is uniformly distributed over $[0,n)$
    \Require $U_m$ is uniformly distributed over $[0,m)$
    \Require $U_n$ and $U_m$ are independently distributed
    \Ensure $nm$ is $n * m$
    \Ensure $U_{nm}$ is uniformly distributed over $[0,nm)$
\Procedure{combine}{$U_n, n, U_m, m$} 
  \State $U_{nm} \gets U_n * m + U_m$
  \State $nm \gets n * m$
  \State \Return $U_{nm}, nm$
\EndProcedure
\end{algorithmic}
\end{algorithm}

\begin{lemma}
In Algorithm \ref{alg:combine}, $U_{nm}$ is uniformly distributed over $[0,nm)$.
\label{lem:combine}
\end{lemma}

\begin{proof}
Let $X \sim Uniform \{0 ... n-1\}$ and $Y \sim Uniform\{0 ... m-1\}$ be independent uniformly distributed random variables. The joint distribution $(X,Y)$ is uniformly distributed with $nm$ elements of probability $\frac{1}{nm}$. Let $Z$ be the distribution defined as

\begin{equation}
Z = f_{combine}(X,Y) = mX+Y
\end{equation}

The mapping $f_{combine}$ is a bijection between the pair $X \times Y$ and $Z$, so $Z$ is also uniformly distributed and 

\begin{equation}
Z \sim Uniform \{0 ... nm-1\}
\end{equation}
\end{proof}

Algorithm \ref{alg:divide} is the inverse of Algorithm \ref{alg:combine}, allowing us to factorise a uniformly distributed integer into two. For this, the sizes of the output distributions must divide the size of the input distribution.

\begin{algorithm}
\caption{Division of uniformly distributed integers}
\label{alg:divide}
\begin{algorithmic}[1]
    \Require $nm$, $n$, $U_{nm}$ are integers
    \Require $nm>0$, $m>0$
    \Require $nm$ is divisible by $n$
    \Require $U_{mn}$ is uniformly distributed over $[0,nm)$
    \Ensure $n * m = nm$
    \Ensure $U_{n}$ is uniformly distributed over $[0,n)$
    \Ensure $U_{m}$ is uniformly distributed over $[0,m)$
    \Ensure $U_n$ and $U_m$ are independent
\Procedure{divide}{$U_{nm}, mn, n$} 
  \State $U_m \gets U_{nm} \operatorname{div} n$
  \State $U_{n} \gets U_{nm} \mod n$
  \State $m \gets nm / n$
  \State \Return $U_n, U_m, m$
\EndProcedure
\end{algorithmic}
\end{algorithm}

\begin{lemma}
In Algorithm \ref{alg:divide}, $U_n$ is uniformly distributed over $[0,n)$ and $U_m$ is uniformly distributed over $[0,m)$. $U_m$ and $U_n$ are independent.

\label{lem:divide}
\end{lemma}

\begin{proof} $f_{combine}$ is a bijection so has an inverse function 
\begin{equation}    
X = \lfloor Z/n \rfloor, Y = Z \mod n
\end{equation}

Since $X$ and $Y$ are the original random variables, they are independent and uniformly distributed.
\end{proof}

Algorithm \ref{alg:resample} converts a uniformly distributed integer to a smaller range, and also returns a biassed bit. Unlike Algorithm \ref{alg:combine} and Algorithm \ref{alg:divide}, the size of the output distribution depends on the value. The biassed bit also contains entropy, and in fact the total entropy returned by this algorithm is the same as its input entropy.

\begin{algorithm}
\caption{Resampling uniformly distributed integers}
\label{alg:resample}
\begin{algorithmic}[1]
    \Require $U_{n}$, $m$ and $n$ are integers 
    \Require $0 \le m \le n$
    \Require $U_{n}$ is uniformly distributed over $[0,n)$
\Ensure $U_{x}$ is uniformly distributed over $[0,x)$
\Ensure $x = m$ or $x=n-m$
\Ensure $B$ is a Boolean value Bernoulli distributed with $p=\frac{m}{n}$
\Ensure $U_x$ and $B$ are independent
\Procedure{resample}{$U_n, n, m$} 
  \If{$U_n < n$}
    \State $B \gets True$  
    \State $x \gets m$
    \State $U_x \gets U_n$
  \Else
    \State $B \gets False$  
    \State $x \gets n-m$
    \State $U_x \gets U_n-m$
  \EndIf
  \State \Return $U_x, x, B$
\EndProcedure
\end{algorithmic}
\end{algorithm}

\begin{lemma}
In Algorithm \ref{alg:resample}, $U_x$ is uniformly distributed over $[0,x)$.
\label{lem:resample}
\end{lemma}

\begin{proof}
    XX define $f_resample(X) \rightarrow Y, Z$ where

    Let $y = U_n$.
    $P(y<m) = \frac{m}{n}$ so $B \sim Bernoulli\{\frac{m}{n}\}$.

If $y < m$, then $y$ is uniformly distributed between $[0,m)$.

If $y \ge m$, then $y$ is uniformly distributed between $[m, n)$, so $y-m$ is uniformly distributed between $[0, n-m)$.
\end{proof}

\begin{lemma}
\label{lem:conservation}
Algorithms \ref{alg:combine}-\ref{alg:resample} conserve entropy.
\end{lemma}

\begin{proof}
For Algorithm \ref{alg:combine}, we see that the initial entropy $H_{lhs}$ and the final entropy $H_{rhs}$ are equal.

\begin{equation}
H_{lhs} = \log_2n + \log_2m = \log_2nm = H_{rhs}
\end{equation}

The same proof applies to Algorithm \ref{alg:divide}. For Algorithm \ref{alg:resample} we need to look at the expected entropy $\mathbb{E}(H)$ before and after the operation:

\begin{align}
H_{lhs}             = &\log_2n \\
\mathbb{E}(H_{rhs}) = & \mathbb{E}(H(B) + H(U_x)) \\
                    = & \mathbb{E}(H(B)) + \mathbb{E}(H(U_x)) \\
                    = & -p\log_2p - (1-p)\log_2(1-p) + p\log_2m + (1-p)\log_2(n-m) \\
                    = & -\frac{m}{n}\log_2\frac{m}{n} - \frac{n-m}{n}\log_2\frac{n-m}{n} + \frac{m}{n}\log_2m + \frac{n-m}{n}\log_2(n-m) \\
                    = &(-\frac{m}{n}\log_2m + -\frac{m}{n}\log_2n) \\
                    & + (- \frac{n-m}{n}\log_2(n-m) + \frac{n-m}{n}\log_2n)\\
                    &+ \frac{m}{n}\log_2m + \frac{n-m}{n}\log_2(n-m) \\
                     = & \log_2n
\end{align}

\end{proof}

Of course, entropy conversion algorithms which rely on resampling \em do \em lose entropy, and this can be quantified by the $B$ term in Algorithm \ref{alg:resample}. Every time an algorithm takes a decision, entropy is lost.

\subsection{Generating uniform integers}

Algorithm \ref{alg:generate-uniform} reads binary entropy from a $fetch()$ function, and outputs a uniform integer in the range $[0,n)$. The algorithm makes use of an \em entropy store \em $U_s$ which is carried over in between function calls. In a practial implementation, $U_s$ and $s$ can be captured variables or class members. Initially the entropy store is empty (containing $0$ entropy) with $U_s = 0$ and $s=1$.

The overall strategy of Algorithm \ref{alg:generate-uniform} is to use $resample$ (Algorithm \ref{alg:resample}) to ensure that $s$ is a multiple of $n$, then use the $divide$ algorithm (Algorithm \ref{alg:divide}) to divide $U_s$ into $U_n$. $U_n$ is returned as the result, and $U_s$ which is stored for the next invocation. The calculation is structured so that when $s$ is large, the entropy lost by $resample$ is very small.

The $resample$ on line 6 resizes $s$ to a multiple of $n$. It is overwhelmingly likely that $b$ is false because $n$ is much smaller than $s$, so we can then proceed to line 8 where we divide $U_s$ into $U_n$ and the new $U_s$. On line 9, return $U_n$ as the result and $U_s$ and $s$ as the input to the next invocation.

An C implemetation of Algorithm \ref{alg:generate-uniform} is given in Appendix \ref{adx:source}.

\begin{algorithm}
\caption{Generating uniformly distributed integers}
\label{alg:generate-uniform}
\begin{algorithmic}[1]
\Require Integers $0 < n\le N$
\Require $fetch()$ returns Bernoulli entropy with $p=0.5$
\Require $U_s$ is uniformly distributed over $[0,s)$
\Ensure $U_n$ is uniformly distributed over $[0,n)$
\Ensure $U_s$ is uniformly distributed over $[0,s)$
\Procedure{generate\_uniform}{$U_s, s, n, N$} 
  \While {True}
    \While {$s < N$}
        \State $U_s, s \gets combine(U_s, s, fetch(), 2)$
    \EndWhile
    \State $U_s, s, b \gets resample(U_s, s, s \mod n)$ 
    \If{$ \neg b$}
        \State $U_n, U_s, s \gets divide(U_s, s, n)$
        \State \Return $U_s, s, U_n$
    \EndIf
  \EndWhile
\EndProcedure
\end{algorithmic}
\end{algorithm}

\begin{lemma}
    In Algorithm \ref{alg:generate-uniform}, 
$U_n$ is uniformly distributed over $[0,n)$ and 
$U_s$ is uniformly distributed over $[0,s)$.
\end{lemma}

\begin{proof}
The values $U_n$, $n$, $U_s$ and $s$ have been generated by Algorithms \ref{alg:combine}, \ref{alg:divide} and \ref{alg:resample}. By Lemmas \ref{lem:combine}, \ref{lem:divide} and \ref{lem:resample}, $U_n$ and $U_s$ are uniformly distributed.
\end{proof}

Algorithm \ref{alg:generate-uniform} terminates with probability 1.

\begin{lemma}
    \label{lem:shannon-inequality}

For $p,q \in \mathbb{R}$, where $0 \le p\le q \le 0.5$, 

\begin{equation}
-p\log_2 p - (1-p)\log_2(1-p) \le -q\log_2 q - (1-q)\log_2(1-q)
\end{equation}
\end{lemma}

\begin{proof}
    Let
    \begin{align}
        g(p) & = -p\log_2 p - (1-p)\log_2(1-p) \\
        \implies g'(p) & = \log_2\frac{1-p}{p} = \log_2(\frac{1}{p}-1) \ge \log_21 = 0 
    \end{align}
Since the derivative of $g>0$ it means that $g$ is monotonic.
\end{proof}

\begin{definition}
    Let $\epsilon = \epsilon(p)$ be the expected entropy loss function of Algorithm \ref{alg:generate-uniform}, where $p=\frac{n-1}{N}$.
\end{definition}

\begin{theorem}
    \label{thm:loss}
If $p = \frac{n-1}{N} < 0.5$,

\begin{equation}
0 \le \epsilon(p) \le -\frac{p}{1-p}\log_2p - \log_2(1-p)
\end{equation}

\end{theorem}

\begin{proof}
For each iteration $i$ of Algorithm \ref{alg:generate-uniform}, let $p_i = \frac{s_i \mod n}{s_i}$. But $(s_i \mod n) \le n-1$ and $s_i \ge N$, so $\frac{s_i \mod n}{s_i} \le \frac{n-1}{N}$, so $p_i \le p$. On each iteration, the entropy lost is equal to the entropy in the variable $b_i \sim Bernoulli\{p_i\}$, which is given by the entropy equation for a Bernoulli distribution \ref{todo}:

\begin{equation}
H(b_i) = -p_i\log_2p_i - (1-p_i)\log_2(1-p_i)
\end{equation}

Therefore, 

\begin{equation}
0 \le H(b_i) \le -p\log_2p - (1-p)\log_2(1-p) 
\end{equation}


by Lemma \ref{lem:shannon-inequality}. The expected number of iterations $N$ is given by

\begin{align}
& N = 1 + p_iN \le 1 + pN \\
\implies & N-pN \le 1 \\
\implies & N(1-p) \le 1 \\
\implies & N \le \frac{1}{1-p}
\end{align}

The total entropy lost by the algorithm is given by the number of iterations of the algorithm multiplied by the entropy lost in each iteration.

\begin{align}
0 \le NH(b_i) \le & \frac{1}{1-p}(-p\log_2p - (1-p)\log_2(1-p) ) \\
= & -\frac{p}{1-p}\log_2p - \log_2(1-p)
\end{align}

We can also end up in the situation where $(s \mod n) = 0$ already, in which the \em resample \em step always succeeds with no entropy loss, so $\epsilon=0$.
\end{proof}

The actual entropy loss incurred by Algorithm \ref{alg:generate-uniform} depends on whatever values are found in $U_s$ and $s$, so we can only give an upper bound.

\begin{corollary}
The entropy efficiency $\eta$ of Algorithm \ref{alg:generate-uniform} is bounded by

\begin{equation}
\frac{\log_2n}{\log_2n + \epsilon(\frac{n-1}{N})} \le \eta \le 1
\label{eq:generate-uniform-efficiency}
\end{equation}
\end{corollary}

\begin{proof}
\begin{align}
    \eta & = \frac{H_{out}}{H_{in}} \\
         & = \frac{H_{out}}{H_{out}+\epsilon} \\
         & = \frac{\log_2n}{\log_2n + \epsilon(\frac{n-1}{N})}
\end{align}
Therefore 
\begin{equation}
\frac{\log_2n}{\log_2n + \epsilon(\frac{n-1}{N})} \le \eta \le 1
\end{equation}
from Theorem \ref{thm:loss}.
\end{proof}

To illustrate Equation \ref{eq:generate-uniform-efficiency}, if $N=2^{31}$ and $n=6$, then $\eta \ge 0.99999997$. This means that even with a modest entropy buffer, we can get very good entropy efficiency.

\begin{corollary}
The entropy efficiency $\eta$ of Algorithm \ref{alg:generate-uniform} is arbitrarily close to 1.
\end{corollary}

\begin{proof}
$\epsilon(\frac{n-1}{N}) \rightarrow 0$ as $N \rightarrow \infty$, using L'H\^opital's Rule. Therefore $\eta \rightarrow 1$ as $N \rightarrow \infty$.
\end{proof}






\subsection{Generating weighted integer distributions}

Algorithm \ref{alg:generate-distribution} shows how we can generate an arbitrary  distribution of $k$ outcomes where each outcome has integer weight $\{w_0, w_1, ..., w_{k-1}\}$, normalised to a discrete distribution with probabilities $\{\frac{w_0}{n}, \frac{w_1}{n}, ..., \frac{w_{k-1}}{n}\}$ where $n=\sum_i w_i$ is the total weight.

The algorithm works by constructing a surjective mapping from the $n$ outcomes of a uniform distribution to the $k$ outputs of the weighted distribution. This uses a simple lookup table of size $n$, constructed using Algorithm \ref{alg:generate-lookup-tables}. If $n$ is very large we could implement the mapping as a binary search on $offset$ instead. To generate an output, call $generate\_uniform$ to generate a integer in the range $[0,n)$, and use the number in the $output$ lookup table as the result. 

We would expect this process to lose entropy because in general the output distribution contains less entropy than the input distribution. The new idea is that we can recover all of the entropy from this process. By splitting up $X$ into chunks of size $w_i$, we see that each chunk contains a uniform distribution of size $w_i$ as illutrated in Figure \ref{fig:chunks}. The chunk contains $\log_2w_i$ entropy. We can use $combine$ to return this entropy back to the entropy store.

Lemma \ref{lem:distribution-conservation} shows us that when we account for this extra entropy then no entropy is lost.

\begin{definition}
    Let $f_{distribute}: X \rightarrow Y, Z$ be the function that maps random variable $X \sim Uniform\{n\}$ to independent random variables $Y$ and $Z \sim Uniform\{w_i\}$.

    \begin{align}
    Y &= max(j : \sum_{i<j}w_i<X) \\
    Z &= X - \sum_{i<Y}w_i
    \end{align}

\end{definition}

Note that $f_{distribute}$ is a generalisation of XX $f_{combine}$ and $f_{sample}$ XX we didn't define $f_{sample}$.

\begin{lemma}
    \label{lem:distribution-conservation}
    $f_{distribute}$ does not lose entropy
\end{lemma}

\begin{proof}
    \begin{align}
    \mathbb{E}(H(X)) & = \log_2 n \\
    \mathbb{E}(H(Y) + H(Z)) &=  \mathbb{E}(H(Y)) + \mathbb{E}(H(Z)) \\
               & = - \sum_i p_i \log_2p_i + \sum_i p_iH(Y|X=i) \\
               & = - \sum_i \frac{w_i}{n} \log_2 \frac{w_i}{n} + \sum_i \frac{w_i}{n}\log_2 w_i \\
               & = - \sum_i \frac{w_i}{n}(\log_2 w_i - \log_2 n) + \sum_i \frac{w_i}{n}\log_2 w_i \\
               & = - \sum_i \frac{w_i}{n}\log_2 w_i + \sum_i \frac{w_i}{n} \log_2 n + \sum_i \frac{w_i}{n}\log_2 w_i \\
               & = \sum_i \frac{w_i}{n} \log_2 n \\
               & = \frac{\log_2 n}{n} \sum_i w_i \\
               & = \frac{\log_2 n}{n} n \\
               & = \log_2 n
    \end{align}
\end{proof}

\begin{algorithm}
\caption{Constructing lookup tables for a weighted random variable}
\label{alg:generate-lookup-tables}
\begin{algorithmic}[1]
\Require $weights$ is a list of integers $\ge0$
\Procedure{make\_distribution}{$weights$} 
  \State $outputs \gets []$
  \State $offsets \gets []$
  \For {$w_i \in weights$}
    \State $offsets \gets offsets + [|outputs|]$
    \State $outputs \gets outputs + [i] * w_i$
  \EndFor
  \State \Return $outputs, offsets$
\EndProcedure
\end{algorithmic}
\end{algorithm}


\begin{algorithm}
\caption{Generating a weighted random variable}
\label{alg:generate-distribution}
\begin{algorithmic}[1]
\Require $U_s$, $s$, $N$ are integers
\Require $weights$ is an array of integers $w_i \ge 0$
\Require $offset$ generated by $make\_distribution$ 
\Require $outputs$ generated by $make\_distribution$
\Require $U_s$ is uniformly distributed over $[0,s)$
\Require $N >> |outputs|$
\Ensure $U_s$ is uniformly distributed over $[0,s)$
\Procedure{generate\_distribution}{$U_s, s, N, weights, outputs, offsets$} 
    \State $n \gets |outputs|$
    \State $U_s, s, U_n \gets generate\_uniform(U_s, s, N, n)$
    \State $U_s, s = combine(U_s, s, U_n - offsets[U_n], weights[outputs[U_n]])$
    \State \Return $U_s, s, outputs[U_n]$
\EndProcedure
\end{algorithmic}
\end{algorithm}


\begin{lemma}
The inverse of $f_{distribute}$ is given by

    \begin{equation}
    f^{-1}_{distribute}(Y,Z) = \sum_{i<Y}w_i + Z
    \end{equation}
\end{lemma}

\begin{proof}
    Show that $f^{-1}f= I$

\end{proof}

\subsection {Extracting entropy from weighted integer distributions}

Algorithm \ref{alg:combine-distribution} performs the inverse of Algorithm \ref{alg:generate-distribution}. It reconstructs a uniform distribution $X$ of size $n$ from the weighted random variable $Y$, and combines it with a uniform distribution $Z$ of size $w_i$. This allows the entropy store to absorb the entropy from $i$.

Note that when performing the inverse of Algorithm \ref{alg:generate-distribution}, we don't need to supply it the *same* random variable $U_x$ that was originally used, and we can just get one from our entropy store.

We'll ignore potential integer overflows of $U_s$ and $s$.

\begin{algorithm}
\caption{Extracting entropy from a weighted random variable}
\label{alg:combine-distribution}
\begin{algorithmic}[1]
\Require $U_s$, $s$, $N$ are integers
\Require $weights$ is an array of integers $w_i \ge 0$
\Require $offset$ generated by $make\_distribution$ 
\Require $outputs$ generated by $make\_distribution$
\Require $U_s$ is uniformly distributed over $[0,s)$
\Require $N >> |outputs|$
\Ensure $U_s$ is uniformly distributed over $[0,s)$
\Procedure{combine\_distribution}{$U_s, s, N, weights, outputs, offsets, i$} 
    \State $U_s, s, U_x \gets generate\_uniform(U_s, s, N, weights[i])$
    \State $U_s, s \gets combine(U_s, s, U_x + offsets[i], |outputs|)$
    \State \Return $U_s, s$
\EndProcedure
\end{algorithmic}
\end{algorithm}



\begin{lemma}
    \label{lem:hloss_monotonic}
    $\epsilon(p)$ is monotonically increasing in the range $0 < p < 1$.
\end{lemma}

\begin{proof}The derivative of $\epsilon(p)$ is
    \begin{equation}
        \frac{-\log_2p}{(1-p)^2}
    \end{equation}
    which is positive in the range $0 < p < 1$.
\end{proof}

\begin{lemma}
The entropy loss of Algorithm \ref{alg:combine-distribution} is no more than than $\epsilon(\frac{n-1}{N})$
\end{lemma}

\begin{proof}
    The entropy-losing step in Algorithm \ref{alg:combine-distribution} is $generate\_uniform$ which loses $\epsilon(\frac{x-1}{N})$ bits. $x \le n$, so from Lemma \ref{lem:hloss_monotonic}, this means that $\epsilon(\frac{x-1}{N}) \le \epsilon(\frac{n-1}{N})$.
\end{proof}

To create a generic entropy converter between weighted distributions, we can use Algorithm \ref{alg:generate-distribution} to generate the output distribution, but replace the $fetch$ step in Algorithm \ref{alg:generate-uniform} with a call to $combine\_distribution$.





\section {Evaluation}

We have already shown that the ES algorithms are more efficient than optimal classical algorithms such as Knuth-Yao and the Interval Algorithm, but it is still instructive to visualise this comparison.

Figure \ref{fig:uniform-losses} shows the calculated entropy loss for generating uniform integers, comparing ES methods with unbatched von Neumann (vN) and Knuth-Yao samplers. These graphs show exact calculation unless stated otherwise. We can observe that for vN and KY, the entropy loss depends on the number being generated. KY loses up to 2 bits per output. ES is not dependent on the generated integer, but loses efficiency at high $n$ as it approaches its minimum store size $N$. We also show real entropy loss by simulating the store, which depends on the previous size of the store, and is within the worst case bound $\epsilon$. As expected, using a larger entropy buffer yields greater efficiency, but would be at the cost of a larger pre-fetch of entropy and slower divmod operations. \cite{inteldivmodslowness}

\begin{figure}[ht]
\centering
\includegraphics[width=0.8\textwidth]{uniform_losses.png}
\caption{Entropy losses for uniform integer generation.}
\label{fig:uniform-losses}
\end{figure}

Figure \ref{fig:shuffling-efficiency} shows the overall impact of entropy losses when applied to the shuffling a deck of $n$ cards using the Fisher-Yates algorithm \cite{fisheryates}. As an example, we can calculate that shuffling a deck of 52 cards can be done with an entropy loss of XX, or an entropy efficiency of XX.

\begin{figure}[ht]
\centering
\includegraphics[width=0.8\textwidth]{shuffling_efficiency.png}
\caption{Entropy efficiency shuffling cards.}
\label{fig:shuffling-efficiency}
\end{figure}

When generating Bernoulli variables, we see in Figure \ref{fig:bernoulli-efficiency} that the entropy efficiency of IA drops off significantly. IA must fetch between 1 and 2 bits per output on average. By contrast, ES algorithms to not necessarily fetch any bits to generate an output, as there may be enough entropy in the store already, so ES just shrinks the size of its store to generate an output. Figure \ref{fig:bernoulli-rate} shows that IA cannot increase its output rate as the entropy of the output distribution drops.

\begin{figure}[ht]
\centering
\includegraphics[width=0.8\textwidth]{bernoulli_efficiency.png}
\caption{Entropy efficiency generating Bernoulli distributions.}
\label{fig:bernoulli-efficiency}
\end{figure}

\begin{figure}[ht]
\centering
\includegraphics[width=0.8\textwidth]{bernoulli_rate.png}
\caption{Output rate for Bernoulli distributions.}
\label{fig:bernoulli-rate}
\end{figure}

We can also evaluate ES algorithms in terms of memory, speed, setup costs and flexibility. ES can be written to use 2 integer divmod operations (see \ref{appendix}), which are likely to prove a bottleneck compared to say Lemire's algorithm \ref{lemire}, or table-based algorithms like FLDR \ref{}. Integer division and multiplication is $O(m \log m)$ \ref{integer multiplication}, for an $m$ bit buffer the complexity of ES is $O(m \log m)$, but for practical purposes it is $O(1)$ in time and space.

ES is able to generate perfect distributions and does not have (albeit minimal) truncation errors found in IA.

\section{Conclusion}

We have introduced a new class of algorithm to convert entropy via a uniform entropy store efficiently, and by caching unused entropy between outputs we can achieve arbitrarily low entropy losses.

We have shown that the entropy efficiency limits of classical algorithms like Knuth-Yao and the Interval Algorithm can be avoided by using an entropy store. 

The ES algorithms are simple and practical enough to be of use in low entropy regimes where the availability of entropy is a bottleneck. The difference with classical algorithms is particularly marked on Bernoulli outputs.

?? Replacing division with something else ??

\printbibliography

\section {Appendix A}
Source code for $generate\_uniform$, written in C.

\begin{verbatim}
    const uint32_t N = 1<<31;
    uint32_t s_value = 0, s_range = 1;

    uint32_t generate_uniform(uint32_t n)
    {
        for(;;)
        {
            // Preload entropy one bit at a time into s
            while(s_range < N)
            {
                s_value <<= 1;
                s_value |= fetch();
                s_range <<= 1;
            }
            // Resample entropy s to a multiple of n
            uint32_t r = s_range / n;
            uint32_t c = s_range % n;
            if(s_value >= c)
            {
                // Resample successful
                s_value -= c;
                uint32_t a = s_value / n;
                uint32_t b = s_value % n;
                s_value = a;
                s_range = r; 
                return b;
            }
            else
            {
                // Resample unsuccessful
                s_range = c;
            }
        }
    }
\end{verbatim}

\end{document}
